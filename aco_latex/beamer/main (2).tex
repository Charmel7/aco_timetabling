\documentclass[aspectratio=169,12pt]{beamer}

% ============================================
% PACKAGES ET CONFIGURATION
% ============================================
\usepackage[utf8]{inputenc}
\usepackage[T1]{fontenc}
\usepackage[french]{babel}
\usepackage{amsmath,amssymb,amsthm}
\usepackage{graphicx}
\usepackage{tikz}
\usetikzlibrary{arrows.meta,shapes,positioning,shadows,calc}
\usepackage{algorithm}
\usepackage{algpseudocode}
\usepackage{booktabs}
\usepackage{colortbl}
\usepackage{xcolor}

% ============================================
% THÈME FUTURISTE PERSONNALISÉ
% ============================================
\usetheme{Madrid}
\usecolortheme{default}

% Couleurs futuristes
\definecolor{cyandark}{RGB}{0,150,200}
\definecolor{cyanlight}{RGB}{100,200,255}
\definecolor{violetdeep}{RGB}{75,0,130}
\definecolor{violetlight}{RGB}{138,43,226}
\definecolor{grisanthracite}{RGB}{40,40,45}
\definecolor{grisclair}{RGB}{220,220,230}
\definecolor{neonpink}{RGB}{255,0,127}
\definecolor{neongreen}{RGB}{57,255,20}

% Configuration des couleurs du thème
\setbeamercolor{structure}{fg=cyandark}
\setbeamercolor{palette primary}{bg=grisanthracite,fg=white}
\setbeamercolor{palette secondary}{bg=violetdeep,fg=white}
\setbeamercolor{palette tertiary}{bg=cyandark,fg=white}
\setbeamercolor{palette quaternary}{bg=violetlight,fg=white}
\setbeamercolor{frametitle}{bg=grisanthracite,fg=cyanlight}
\setbeamercolor{title}{fg=cyanlight}
\setbeamercolor{block title}{bg=cyandark,fg=white}
\setbeamercolor{block body}{bg=grisclair,fg=black}
\setbeamercolor{block title alerted}{bg=neonpink,fg=white}
\setbeamercolor{block body alerted}{bg=grisclair,fg=black}
\setbeamercolor{block title example}{bg=violetlight,fg=white}
\setbeamercolor{block body example}{bg=grisclair,fg=black}

% Police moderne
\usefonttheme{professionalfonts}
\setbeamerfont{title}{size=\huge,series=\bfseries}
\setbeamerfont{frametitle}{size=\Large,series=\bfseries}
\setbeamerfont{block title}{size=\large,series=\bfseries}

% Templates personnalisés
\setbeamertemplate{navigation symbols}{}
\setbeamertemplate{footline}[frame number]
\setbeamertemplate{itemize items}[circle]
\setbeamertemplate{sections/subsections in toc}[circle]

% Fond avec dégradé subtil
\setbeamertemplate{background}{
  \begin{tikzpicture}[remember picture,overlay]
    \fill[left color=white,right color=grisclair!30] 
      (current page.south west) rectangle (current page.north east);
  \end{tikzpicture}
}

% ============================================
% INFORMATIONS DE LA PRÉSENTATION
% ============================================
\title[ACO pour Timetabling]{Exploration de l'Algorithme d'Optimisation\\par Colonies de Fourmis (ACO)}
\subtitle{Gestion des Groupes Pédagogiques\\Campus de Sogbo-Aliho}
\author{Groupe 2}
\institute{
  UNIVERSITÉ NATIONALE DES SCIENCES, TECHNOLOGIES,\\
  INGÉNIERIE ET MATHÉMATIQUES (UNSTIM)\\
  \vspace{0.3cm}
  École Nationale Supérieure de Génie\\
  Mathématique et Modélisation (ENSGMM)
}
\date{Année Académique 2025-2026}

% ============================================
% DÉBUT DU DOCUMENT
% ============================================
\begin{document}

% ====== SLIDE 1 : PAGE DE TITRE ======
\begin{frame}[plain]
  \begin{tikzpicture}[remember picture,overlay]
    \fill[left color=violetdeep,right color=cyandark] 
      (current page.south west) rectangle (current page.north east);
    \node[white,opacity=0.1] at (current page.center) 
      {\includegraphics[width=0.5\paperwidth]{example-image}};
  \end{tikzpicture}
  \titlepage
\end{frame}

% ====== SLIDE 2 : CONTEXTE ======
\begin{frame}{Contexte du Problème}
  \begin{columns}[T]
    \column{0.5\textwidth}
    \begin{alertblock}{Défi Majeur}
      Affectation optimale des classes aux salles sur le campus de Sogbo-Aliho
    \end{alertblock}
    
    \vspace{0.5cm}
    \textbf{Problématique :}
    \begin{itemize}
      \item Ensemble de classes avec emplois du temps
      \item Salles avec capacités limitées
      \item Minimiser les conflits temporels
      \item Respecter contraintes de capacité
    \end{itemize}
    
    \column{0.5\textwidth}
    \begin{tikzpicture}[scale=0.8]
      % Illustration schématique
      \foreach \i in {1,2,3} {
        \draw[fill=cyanlight!50,draw=cyandark,thick] 
          (0,\i*1.2) rectangle (2,\i*1.2+0.8);
        \node at (1,\i*1.2+0.4) {\footnotesize Classe \i};
      }
      \draw[->,ultra thick,violetlight] (2.5,2.4) -- (3.5,2.4);
      \foreach \j in {1,2} {
        \draw[fill=violetlight!50,draw=violetdeep,thick] 
          (4,\j*1.5) rectangle (6,\j*1.5+1);
        \node at (5,\j*1.5+0.5) {\footnotesize Salle \j};
      }
    \end{tikzpicture}
  \end{columns}
  
  \vspace{0.5cm}
  \begin{exampleblock}{Classification}
    \textbf{University Timetabling Problem} — NP-difficile
  \end{exampleblock}
\end{frame}

% ====== SLIDE 3 : PLAN ======
\begin{frame}{Plan de la Présentation}
  \tableofcontents
\end{frame}

% ============================================
% SECTION 1 : PRINCIPE BIOLOGIQUE
% ============================================
\section{Principe Biologique des Fourmis}

% ====== SLIDE 4 : COMPORTEMENT NATUREL ======
\begin{frame}{Comportement Naturel des Fourmis}
  \begin{columns}
    \column{0.55\textwidth}
    \begin{block}{Intelligence Collective}
      Agents simples résolvant collectivement des problèmes complexes sans coordination centralisée
    \end{block}
    
    \vspace{0.3cm}
    \textbf{Mécanisme clé : La Phéromone}
    \begin{itemize}
      \item Substance chimique déposée sur le chemin
      \item Sécrétée par glandes exocrines
      \item Guide les autres fourmis
      \item Crée une boucle de rétroaction positive
    \end{itemize}
    
    \column{0.45\textwidth}
    \begin{tikzpicture}[scale=0.9]
      % Nid
      \node[circle,fill=violetlight!70,minimum size=1cm] (nest) at (0,0) {\tiny Nid};
      % Nourriture
      \node[star,star points=5,fill=neongreen!70,minimum size=1cm] (food) at (6,0) {\tiny Food};
      % Chemin avec phéromones
      \draw[->,thick,cyandark,decorate,decoration={snake,amplitude=0.5mm}] 
        (nest) to[bend left=30] node[above,sloped] {\tiny phéromone} (food);
      \draw[->,thick,cyanlight] 
        (nest) to[bend right=30] (food);
      % Fourmis
      \foreach \x in {1,2,...,5} {
        \fill[black] (\x,{0.3*sin(\x*60)}) circle (1.5pt);
      }
    \end{tikzpicture}
  \end{columns}
\end{frame}

% ====== SLIDE 5 : EXPÉRIENCE DU PONT DOUBLE ======
\begin{frame}{L'Expérience du Pont Double (Goss et al., 1989)}
  \begin{columns}
    \column{0.5\textwidth}
    \textbf{Dispositif :}
    \begin{itemize}
      \item Colonie séparée de la nourriture
      \item Deux branches de longueurs différentes
    \end{itemize}
    
    \vspace{0.5cm}
    \textbf{Phases observées :}
    \begin{enumerate}
      \item \textcolor{violetdeep}{Exploration aléatoire}
      \item \textcolor{cyandark}{Accumulation sur branche courte}
      \item \textcolor{neongreen}{Convergence vers l'optimum}
    \end{enumerate}
    
    \column{0.5\textwidth}
    \begin{tikzpicture}[scale=1]
      \node[circle,fill=violetlight,minimum size=8mm] (A) at (0,0) {N};
      \node[circle,fill=neongreen,minimum size=8mm] (B) at (5,0) {F};
      
      % Chemin long
      \draw[thick,gray!40,line width=2pt] (A) to[bend left=40] 
        node[above,sloped] {\tiny Chemin long} (B);
      % Chemin court avec plus de phéromones
      \draw[thick,cyandark,line width=5pt] (A) to[bend right=20] 
        node[below,sloped] {\tiny Chemin court} (B);
      
      \node[below=0.5cm] at (2.5,-1) {\small Plus de passages $\Rightarrow$ Plus de phéromones};
    \end{tikzpicture}
  \end{columns}
  
  \vspace{0.5cm}
  \begin{exampleblock}{Facteurs Clés}
    \textbf{Évaporation} (oubli des mauvaises décisions) + \textbf{Comportement probabiliste} (exploration continue)
  \end{exampleblock}
\end{frame}

% ====== SLIDE 6 : TRANSPOSITION ALGORITHMIQUE ======
\begin{frame}{Transposition au Domaine Algorithmique}
  \begin{center}
    \begin{tikzpicture}[node distance=0.5cm]
      % Colonne gauche : Système naturel
      \node[fill=cyanlight!30,rounded corners,minimum width=5cm,minimum height=0.8cm] 
        (h1) {\textbf{Système Naturel}};
      \node[below=of h1,fill=grisclair,minimum width=5cm,minimum height=0.6cm] 
        (n1) {Fourmi réelle};
      \node[below=of n1,fill=grisclair,minimum width=5cm,minimum height=0.6cm] 
        (n2) {Phéromone chimique};
      \node[below=of n2,fill=grisclair,minimum width=5cm,minimum height=0.6cm] 
        (n3) {Chemin physique};
      \node[below=of n3,fill=grisclair,minimum width=5cm,minimum height=0.6cm] 
        (n4) {Concentration};
      \node[below=of n4,fill=grisclair,minimum width=5cm,minimum height=0.6cm] 
        (n5) {Évaporation naturelle};
      
      % Colonne droite : Système artificiel
      \node[fill=violetlight!30,rounded corners,minimum width=5cm,minimum height=0.8cm,right=2cm of h1] 
        (h2) {\textbf{Système Artificiel}};
      \node[below=of h2,fill=grisclair,minimum width=5cm,minimum height=0.6cm] 
        (a1) {Fourmi artificielle (agent)};
      \node[below=of a1,fill=grisclair,minimum width=5cm,minimum height=0.6cm] 
        (a2) {Phéromone numérique ($\tau$)};
      \node[below=of a2,fill=grisclair,minimum width=5cm,minimum height=0.6cm] 
        (a3) {Solution candidate};
      \node[below=of a3,fill=grisclair,minimum width=5cm,minimum height=0.6cm] 
        (a4) {Qualité de la solution};
      \node[below=of a4,fill=grisclair,minimum width=5cm,minimum height=0.6cm] 
        (a5) {Facteur d'évaporation $\rho$};
      
      % Flèches
      \foreach \i in {1,...,5} {
        \draw[->,ultra thick,neonpink] (n\i.east) -- (a\i.west);
      }
    \end{tikzpicture}
  \end{center}
\end{frame}

% ============================================
% SECTION 2 : ALGORITHME ACO
% ============================================
\section{Description de l'Algorithme ACO}

% ====== SLIDE 7 : STRUCTURE GÉNÉRALE ======
\begin{frame}{Structure Générale de l'ACO}
  \begin{columns}
    \column{0.5\textwidth}
    \begin{block}{Métaheuristique Itérative}
      Population d'agents construisant des solutions de manière probabiliste
    \end{block}
    
    \vspace{0.3cm}
    \textbf{Composants essentiels :}
    \begin{itemize}
      \item Graphe de construction $G = (C, L)$
      \item Phéromones $\tau_{ij}$
      \item Information heuristique $\eta_{ij}$
      \item Règle de transition probabiliste
      \item Mise à jour des phéromones
    \end{itemize}
    
    \column{0.5\textwidth}
    \begin{tikzpicture}[scale=0.85,every node/.style={font=\footnotesize}]
      \node[draw,rectangle,fill=cyandark!30,rounded corners,minimum width=3.5cm] 
        (init) at (0,0) {Initialisation};
      \node[draw,rectangle,fill=cyanlight!30,rounded corners,minimum width=3.5cm,below=0.5cm of init] 
        (const) {Construction solutions};
      \node[draw,rectangle,fill=violetlight!30,rounded corners,minimum width=3.5cm,below=0.5cm of const] 
        (eval) {Évaluation};
      \node[draw,rectangle,fill=neongreen!30,rounded corners,minimum width=3.5cm,below=0.5cm of eval] 
        (update) {Mise à jour phéromones};
      \node[draw,diamond,fill=neonpink!30,aspect=2,below=0.5cm of update] 
        (stop) {Stop?};
      \node[draw,rectangle,fill=violetdeep!30,rounded corners,text=white,below=0.5cm of stop] 
        (result) {Meilleure solution};
      
      \draw[->,thick] (init) -- (const);
      \draw[->,thick] (const) -- (eval);
      \draw[->,thick] (eval) -- (update);
      \draw[->,thick] (update) -- (stop);
      \draw[->,thick] (stop) -- node[right] {Oui} (result);
      \draw[->,thick] (stop.west) -- ++(-1,0) |- node[left,pos=0.25] {Non} (const.west);
    \end{tikzpicture}
  \end{columns}
\end{frame}

% ====== SLIDE 8 : RÈGLE DE TRANSITION ======
\begin{frame}{Règle de Transition Probabiliste}
  \begin{alertblock}{Formule Fondamentale}
    La probabilité qu'une fourmi $k$ affecte la classe $i$ à la salle $j$ :
    \[
      p^k_{ij} = \frac{[\tau_{ij}]^\alpha \cdot [\eta_{ij}]^\beta}{\sum_{l \in N^k_i} [\tau_{il}]^\alpha \cdot [\eta_{il}]^\beta}
    \]
  \end{alertblock}
  
  \vspace{0.5cm}
  \begin{columns}
    \column{0.5\textwidth}
    \begin{block}{Paramètres}
      \begin{itemize}
        \item $\alpha$ : influence phéromone
        \item $\beta$ : influence heuristique
        \item $N^k_i$ : salles compatibles
      \end{itemize}
    \end{block}
    
    \column{0.5\textwidth}
    \begin{exampleblock}{Équilibre}
      \begin{itemize}
        \item $\alpha$ élevé $\Rightarrow$ \textbf{exploitation}
        \item $\beta$ élevé $\Rightarrow$ \textbf{glouton}
        \item Équilibre optimal crucial
      \end{itemize}
    \end{exampleblock}
  \end{columns}
\end{frame}

% ====== SLIDE 9 : MISE À JOUR PHÉROMONES ======
\begin{frame}{Mise à Jour des Phéromones}
  \begin{columns}
    \column{0.5\textwidth}
    \textbf{1. Évaporation}
    \[
      \tau_{ij} \leftarrow (1-\rho) \cdot \tau_{ij}
    \]
    \begin{itemize}
      \item $\rho \in (0,1]$ : taux d'évaporation
      \item Oublie les mauvaises décisions
      \item Évite la stagnation
    \end{itemize}
    
    \vspace{0.5cm}
    \textbf{2. Dépôt}
    \[
      \tau_{ij} \leftarrow \tau_{ij} + \Delta\tau^{best}_{ij}
    \]
    \begin{itemize}
      \item Seule la meilleure fourmi dépose
      \item Renforce les bonnes solutions
    \end{itemize}
    
    \column{0.5\textwidth}
    \begin{block}{MAX-MIN Ant System}
      Amélioration de l'ACO classique :
      \[
        \tau_{ij} = \max\{\tau_{min}, \min\{\tau_{max}, \tau_{ij}\}\}
      \]
    \end{block}
    
    \vspace{0.3cm}
    \begin{tikzpicture}[scale=0.9]
      \draw[->,thick] (0,0) -- (5,0) node[right] {Itérations};
      \draw[->,thick] (0,0) -- (0,3) node[above] {$\tau$};
      \draw[thick,neonpink] (0,2.5) -- (5,2.5) node[right] {$\tau_{max}$};
      \draw[thick,cyandark] (0,0.5) -- (5,0.5) node[right] {$\tau_{min}$};
      \draw[thick,violetlight,decorate,decoration={snake,amplitude=1mm}] 
        (0,1.5) -- (5,1.8);
      \node[below] at (2.5,-0.3) {\footnotesize Évolution contrôlée};
    \end{tikzpicture}
  \end{columns}
\end{frame}

% ============================================
% SECTION 3 : MODÉLISATION MATHÉMATIQUE
% ============================================
\section{Modélisation Mathématique}

% ====== SLIDE 10 : DÉFINITION FORMELLE ======
\begin{frame}{Définition Formelle du Problème}
  \begin{block}{Données}
    \begin{itemize}
      \item Classes : $C = \{c_1, c_2, \ldots, c_n\}$
      \item Salles : $R = \{r_1, r_2, \ldots, r_k\}$ avec $k < n$
    \end{itemize}
  \end{block}
  
  \begin{alertblock}{Variables de Décision}
    \[
      x_{ij} = \begin{cases}
        1 & \text{si classe } c_i \text{ affectée à salle } r_j \\
        0 & \text{sinon}
      \end{cases}
    \]
    Problème d'optimisation \textbf{discret} : $x_{ij} \in \{0,1\}$
  \end{alertblock}
  
  \begin{exampleblock}{Paramètres}
    \begin{itemize}
      \item $E_i$ : effectif classe $i$
      \item $Cap_j$ : capacité salle $j$
      \item $T_i$ : créneaux horaires classe $i$
      \item $comp_{ij}$ : compatibilité classe-salle
    \end{itemize}
  \end{exampleblock}
\end{frame}

% ====== SLIDE 11 : FONCTION OBJECTIF ======
\begin{frame}{Fonction Objectif}
  \begin{alertblock}{Objectif Principal : Minimiser les Conflits}
    \[
      f_1(x) = \sum_{j=1}^k \sum_{i=1}^n \sum_{h=i+1}^n x_{ij} \cdot x_{hj} \cdot \text{conflict}(c_i, c_h)
    \]
    où $\text{conflict}(c_i, c_h) = 1$ si $T_i \cap T_h \neq \emptyset$
  \end{alertblock}
  
  \vspace{0.3cm}
  \begin{columns}
    \column{0.5\textwidth}
    \textbf{Critères Secondaires :}
    \begin{itemize}
      \item $f_2(x)$ : équilibre charge
      \item $f_3(x)$ : taux d'occupation
    \end{itemize}
    
    \column{0.5\textwidth}
    \begin{block}{Fonction Globale}
      \[
        f(x) = w_1 f_1 + w_2 f_2 + w_3 f_3
      \]
      avec $w_1 \gg w_2, w_3$
    \end{block}
  \end{columns}
  
  \vspace{0.3cm}
  \begin{exampleblock}{Nature du Problème}
    Fonction objectif \textbf{continue} avec variables \textbf{discrètes} $\Rightarrow$ Optimisation combinatoire
  \end{exampleblock}
\end{frame}

% ====== SLIDE 12 : CONTRAINTES ======
\begin{frame}{Contraintes du Problème}
  \begin{columns}
    \column{0.5\textwidth}
    \begin{alertblock}{Contraintes HARD}
      \textbf{Obligatoires} — ne peuvent être violées
      \begin{enumerate}
        \item Unicité d'affectation\\
        $\sum_{j=1}^k x_{ij} = 1 \quad \forall i$
        
        \item Respect capacité\\
        $x_{ij}(E_i - Cap_j) \leq 0$
        
        \item Compatibilité\\
        $x_{ij} \leq comp_{ij}$
        
        \item Non-double occupation
      \end{enumerate}
    \end{alertblock}
    
    \column{0.5\textwidth}
    \begin{block}{Contraintes SOFT}
      \textbf{Préférences} — améliorent la qualité
      \begin{itemize}
        \item Limitation journalière
        \item Éviter séquences longues
        \item Éviter dernière période
        \item Préférences enseignants
        \item Équilibrage charge
      \end{itemize}
      
      \vspace{0.3cm}
      Fonction de pénalité :
      \[
        \text{penalty}(x) = \sum_i w_{S_i} \cdot v_{S_i}(x)
      \]
    \end{block}
  \end{columns}
\end{frame}

% ====== SLIDE 13 : COMPLEXITÉ ======
\begin{frame}{Complexité du Problème}
  \begin{alertblock}{Nature NP-Difficile}
    Le problème de timetabling universitaire est \textbf{NP-difficile}
    \begin{itemize}
      \item Réductible au problème de coloration de graphes (NP-complet)
      \item Espace de recherche : $k^n$ solutions possibles
    \end{itemize}
  \end{alertblock}
  
  \vspace{0.5cm}
  \begin{exampleblock}{Exemple Numérique}
    Pour $n = 30$ classes et $k = 10$ salles :
    \[
      10^{30} \text{ solutions possibles} \approx 10^9 \text{ fois l'âge de l'univers en secondes!}
    \]
    $\Rightarrow$ Énumération exhaustive \textbf{impossible}
  \end{exampleblock}
  
  \vspace{0.5cm}
  \begin{block}{Justification de l'Approche Métaheuristique}
    Les métaheuristiques comme l'ACO offrent un compromis efficace entre \textbf{qualité} et \textbf{temps de calcul}
  \end{block}
\end{frame}

% ============================================
% SECTION 4 : ADAPTATION AU TIMETABLING
% ============================================
\section{Adaptation de l'ACO au Timetabling}

% ====== SLIDE 14 : INFORMATION HEURISTIQUE ======
\begin{frame}{Information Heuristique Adaptée}
  \begin{block}{Composantes Heuristiques}
    L'information $\eta_{ij}$ guide vers des affectations localement bonnes :
  \end{block}
  
  \begin{columns}
    \column{0.5\textwidth}
    \textbf{1. Heuristique de capacité}
    \[
      \eta^{cap}_{ij} = \frac{1}{1 + |Cap_j - E_i|}
    \]
    
    \textbf{2. Heuristique de conflit}
    \[
      \eta^{conf}_{ij} = \frac{1}{1 + \sum_{c_h \in S^k_j} \text{conflict}(c_i, c_h)}
    \]
    
    \column{0.5\textwidth}
    \textbf{3. Heuristique de charge}
    \[
      \eta^{load}_{ij} = \frac{1}{1 + |S^k_j|}
    \]
    
    \textbf{4. Heuristique combinée}
    \[
      \eta_{ij} = \eta^{cap}_{ij} \cdot (\eta^{conf}_{ij})^2 \cdot \eta^{load}_{ij}
    \]
  \end{columns}
  
  \vspace{0.5cm}
  \begin{exampleblock}{Enrichissement}
    Intégration des disponibilités enseignants : $\eta_{ij} \cdot \eta^{avail}_{ij}$
  \end{exampleblock}
\end{frame}

% ====== SLIDE 15 : FORMULE ACO ENRICHIE ======
\begin{frame}{Formule ACO Enrichie pour le Timetabling}
  \begin{alertblock}{Règle de Transition Complète}
    \[
      p^k_{ij} = \frac{[\tau_{ij}]^\alpha \cdot [\eta_{ij}]^\beta \cdot [\mu_j]^\theta \cdot [\lambda_j]^\delta \cdot [\nu_{jh}]^\gamma}{\sum_{l \in N^k_i} [\tau_{il}]^\alpha \cdot [\eta_{il}]^\beta \cdot [\mu_l]^\theta \cdot [\lambda_l]^\delta \cdot [\nu_{lh}]^\gamma}
    \]
  \end{alertblock}
  
  \vspace{0.5cm}
  \begin{columns}
    \column{0.5\textwidth}
    \textbf{Paramètres additionnels :}
    \begin{itemize}
      \item $\mu_j$ : heures théoriques cours
      \item $\lambda_j$ : disponibilité hebdo enseignant
      \item $\nu_{jh}$ : disponibilité créneau $h$
    \end{itemize}
    
    \column{0.5\textwidth}
    \textbf{Exposants de contrôle :}
    \begin{itemize}
      \item $\theta, \delta, \gamma$ : importance relative
      \item Permet d'intégrer préférences
    \end{itemize}
  \end{columns}
  
  \vspace{0.5cm}
  \begin{exampleblock}{Avantage}
    Formulation complète prenant en compte contraintes \textbf{ET} préférences
  \end{exampleblock}
\end{frame}

% ====== SLIDE 16 : SÉLECTION PAR ROULETTE ======
\begin{frame}{Méthode de Sélection par Roulette}
  \begin{columns}
    \column{0.5\textwidth}
    \begin{block}{Principe}
      Sélection proportionnelle aux probabilités $p_i$
      \begin{enumerate}
        \item Calculer probabilités cumulatives
        \item Générer $r \in [0,1]$
        \item Sélectionner $j$ tel que $P_{j-1} < r \leq P_j$
      \end{enumerate}
    \end{block}
    
    \vspace{0.3cm}
    \begin{exampleblock}{Propriété}
      Garantit exploration \textbf{stochastique} de l'espace de recherche
    \end{exampleblock}
    
    \column{0.5\textwidth}
    \begin{tikzpicture}[scale=1]
      % Cercle de roulette
      \draw[thick,fill=grisclair] (0,0) circle (2);
      
      % Secteurs avec différentes tailles
      \fill[cyanlight] (0,0) -- (2,0) arc (0:120:2) -- cycle;
      \fill[violetlight] (0,0) -- (120:2) arc (120:200:2) -- cycle;
      \fill[neongreen!50] (0,0) -- (200:2) arc (200:280:2) -- cycle;
      \fill[neonpink!50] (0,0) -- (280:2) arc (280:360:2) -- cycle;
      
      % Labels
      \node at (60:1.3) {$p_1$};
      \node at (160:1.3) {$p_2$};
      \node at (240:1.3) {$p_3$};
      \node at (320:1.3) {$p_4$};
      
      % Flèche de sélection
      \draw[->,ultra thick,red] (0,0) -- (50:2.5);
      \node[red] at (50:3) {$r$};
    \end{tikzpicture}
  \end{columns}
\end{frame}

% ====== SLIDE 17 : ALGORITHME ACO ADAPTÉ ======
\begin{frame}[fragile]{Algorithme ACO Adapté - Vue d'Ensemble}
  \begin{columns}
    \column{0.5\textwidth}
    \begin{block}{Phase d'Initialisation}
      \begin{itemize}
        \item $\tau_{ij} \leftarrow \tau_{max}$
        \item $S_{best} \leftarrow \emptyset$
        \item $f_{best} \leftarrow +\infty$
      \end{itemize}
    \end{block}
    
    \begin{alertblock}{Boucle Principale}
      Pour chaque fourmi :
      \begin{enumerate}
        \item Construire solution $S^{ant}$
        \item Évaluer $f(S^{ant})$
        \item Mettre à jour si meilleure
      \end{enumerate}
    \end{alertblock}
    
    \column{0.5\textwidth}
    \begin{block}{Construction Solution}
      \begin{itemize}
        \item Parcourir toutes les classes
        \item Identifier salles compatibles
        \item Calculer heuristiques
        \item Sélectionner par roulette
      \end{itemize}
    \end{block}
    
    \begin{exampleblock}{Mise à Jour}
      \begin{itemize}
        \item Évaporation globale
        \item Dépôt meilleure fourmi
        \item Bornes MAX-MIN
      \end{itemize}
    \end{exampleblock}
  \end{columns}
  
  \vspace{0.5cm}
  \centering
  \textbf{Complexité :} $\mathcal{O}(T \cdot m \cdot n \cdot (k+n))$
\end{frame}

% ====== SLIDE 18 : PARAMÈTRES RECOMMANDÉS ======
\begin{frame}{Paramètres Recommandés}
  \begin{center}
    \begin{tabular}{lcc}
      \toprule
      \textbf{Paramètre} & \textbf{Valeur} & \textbf{Rôle} \\
      \midrule
      \rowcolor{grisclair}
      $\alpha$ & 1 & Influence phéromone \\
      $\beta$ & 5 & Influence heuristique (priorité) \\
      \rowcolor{grisclair}
      $\rho$ & 0.01 & Taux d'évaporation (exploration) \\
      $m$ & 30 & Nombre de fourmis \\
      \rowcolor{grisclair}
      $Q$ & 100 & Constante de dépôt \\
      $\tau_{min}$ & 0.01 & Borne inférieure MAX-MIN \\
      \rowcolor{grisclair}
      $\tau_{max}$ & 6 & Borne supérieure MAX-MIN \\
      \midrule
      \rowcolor{cyanlight!30}
      $w_1$ & 1000 & \textbf{Poids conflits (prioritaire)} \\
      \rowcolor{grisclair}
      $w_2$ & 1 & Poids équilibre \\
      $w_3$ & 0.1 & Poids occupation \\
      \bottomrule
    \end{tabular}
  \end{center}
  
  \vspace{0.5cm}
  \begin{block}{Justification}
    $\beta = 5$ : privilégie fortement l'évitement des conflits\\
    $\rho = 0.01$ : faible évaporation favorise l'exploration
  \end{block}
\end{frame}

% ====== SLIDE 19 : COMPLEXITÉ ALGORITHMIQUE ======
\begin{frame}{Complexité Algorithmique}
  \begin{block}{Complexité d'une Itération}
    \textbf{Pour chaque fourmi :}
    \begin{itemize}
      \item Construction : $\mathcal{O}(n \cdot k)$
      \item Évaluation conflits : $\mathcal{O}(n^2)$ pire cas
    \end{itemize}
    
    \textbf{Pour $m$ fourmis :} $\mathcal{O}(m \cdot n \cdot (k+n))$
  \end{block}
  
  \vspace{0.3cm}
  \begin{alertblock}{Complexité Totale}
    Pour $T$ itérations : $\mathcal{O}(T \cdot m \cdot n \cdot (k+n))$
  \end{alertblock}
  
  \vspace{0.5cm}
  \begin{exampleblock}{Exemple Numérique Concret}
    \begin{itemize}
      \item $n = 30$ classes, $k = 10$ salles
      \item $m = 30$ fourmis, $T = 1000$ itérations
      \item $1000 \times 30 \times 30 \times 40 \approx 3.6 \times 10^7$ opérations
      \item \textbf{Temps estimé :} quelques dizaines de secondes
    \end{itemize}
  \end{exampleblock}
\end{frame}

% ============================================
% SECTION 5 : REPRÉSENTATION GRAPHIQUE
% ============================================
\section{Représentation et Visualisation}

% ====== SLIDE 20 : GRAPHE DE CONSTRUCTION ======
\begin{frame}{Représentation Graphique du Problème}
  \begin{columns}
    \column{0.45\textwidth}
    \begin{block}{Graphe Biparti}
      $G = (C \cup R, E)$
      \begin{itemize}
        \item $C$ : classes
        \item $R$ : salles
        \item $E$ : affectations possibles
      \end{itemize}
    \end{block}
    
    \vspace{0.3cm}
    \begin{alertblock}{Phéromones}
      $\tau_{ij}$ sur les arêtes représente la désirabilité d'affecter $c_i$ à $r_j$
    \end{alertblock}
    
    \column{0.55\textwidth}
    \begin{tikzpicture}[scale=0.85]
      % Classes (gauche)
      \foreach \i in {1,2,3,4} {
        \node[circle,draw=cyandark,fill=cyanlight!50,minimum size=8mm] 
          (c\i) at (0,-\i*1.2) {$c_\i$};
      }
      \node[above] at (0,0) {\textbf{Classes}};
      
      % Salles (droite)
      \foreach \j in {1,2,3} {
        \node[rectangle,draw=violetdeep,fill=violetlight!50,minimum size=8mm,rounded corners] 
          (r\j) at (5,-\j*1.5) {$r_\j$};
      }
      \node[above] at (5,0) {\textbf{Salles}};
      
      % Arêtes avec phéromones (épaisseur variable)
      \draw[thick,opacity=0.6] (c1) -- (r1);
      \draw[ultra thick,cyandark] (c1) -- (r2);
      \draw[thick,opacity=0.4] (c2) -- (r1);
      \draw[very thick,violetlight] (c2) -- (r3);
      \draw[ultra thick,neongreen] (c3) -- (r1);
      \draw[thick,opacity=0.5] (c3) -- (r2);
      \draw[thick,opacity=0.6] (c4) -- (r2);
      \draw[very thick,neonpink] (c4) -- (r3);
      
      \node[below,font=\footnotesize] at (2.5,-5.5) 
        {Épaisseur $\propto \tau_{ij}$};
    \end{tikzpicture}
  \end{columns}
\end{frame}

% ====== SLIDE 21 : ILLUSTRATION CONSTRUCTION SOLUTION ======
\begin{frame}{Illustration : Construction d'une Solution}
  \begin{center}
    \begin{tikzpicture}[scale=0.9]
      % Titre étapes
      \node[font=\small] at (0,3.5) {\textbf{Étape 1}};
      \node[font=\small] at (5,3.5) {\textbf{Étape 2}};
      \node[font=\small] at (10,3.5) {\textbf{Étape 3}};
      
      % Étape 1
      \node[circle,fill=cyanlight,minimum size=1cm] (c1a) at (0,2) {$c_1$};
      \node[rectangle,fill=violetlight!30,minimum size=1cm,rounded corners] (r1a) at (1.5,2.5) {$r_1$};
      \node[rectangle,fill=violetlight!30,minimum size=1cm,rounded corners] (r2a) at (1.5,1.5) {$r_2$};
      \draw[->,ultra thick,neongreen] (c1a) -- (r1a);
      
      % Étape 2
      \node[circle,fill=cyanlight,minimum size=1cm] (c2b) at (5,2) {$c_2$};
      \node[rectangle,fill=violetlight!30,minimum size=1cm,rounded corners] (r1b) at (6.5,2.5) {$r_1$};
      \node[rectangle,fill=violetlight!30,minimum size=1cm,rounded corners] (r2b) at (6.5,1.5) {$r_2$};
      \draw[->,thick,opacity=0.3] (c2b) -- (r1b);
      \draw[->,ultra thick,neonpink] (c2b) -- (r2b);
      
      % Étape 3
      \node[circle,fill=cyanlight,minimum size=1cm] (c3c) at (10,2) {$c_3$};
      \node[rectangle,fill=violetlight!30,minimum size=1cm,rounded corners] (r1c) at (11.5,2.5) {$r_1$};
      \node[rectangle,fill=violetlight!30,minimum size=1cm,rounded corners] (r2c) at (11.5,1.5) {$r_2$};
      \draw[->,ultra thick,cyandark] (c3c) -- (r1c);
      
      % Légende
      \node[below,font=\footnotesize] at (5,0.5) 
        {La fourmi construit progressivement une affectation complète};
    \end{tikzpicture}
  \end{center}
  
  \vspace{0.3cm}
  \begin{block}{Mécanisme}
    À chaque étape, la fourmi sélectionne la salle selon $p^k_{ij}$ (phéromones + heuristiques)
  \end{block}
\end{frame}

% ============================================
% SECTION 6 : APPLICATION ET RÉSULTATS
% ============================================
\section{Application au Campus de Sogbo-Aliho}

% ====== SLIDE 22 : CAS D'ÉTUDE ======
\begin{frame}{Cas d'Étude : Campus de Sogbo-Aliho}
  \begin{columns}
    \column{0.5\textwidth}
    \begin{block}{Contexte}
      \begin{itemize}
        \item Institution d'enseignement supérieur
        \item Multiples départements
        \item Ressources limitées
        \item Contraintes complexes
      \end{itemize}
    \end{block}
    
    \vspace{0.3cm}
    \begin{alertblock}{Défis Spécifiques}
      \begin{itemize}
        \item Emplois du temps variables
        \item Capacités hétérogènes
        \item Équipements spécialisés
        \item Préférences enseignants
      \end{itemize}
    \end{alertblock}
    
    \column{0.5\textwidth}
    \begin{exampleblock}{Objectifs}
      \begin{enumerate}
        \item Zéro conflit temporel
        \item Utilisation optimale salles
        \item Satisfaction enseignants
        \item Équilibrage charge
      \end{enumerate}
    \end{exampleblock}
    
    \vspace{0.3cm}
    \begin{tikzpicture}[scale=0.8]
      % Schéma campus simplifié
      \draw[fill=grisclair,rounded corners] (0,0) rectangle (4,2);
      \node at (2,1.5) {\textbf{Campus Sogbo-Aliho}};
      \foreach \x in {0.5,1.5,2.5,3.5} {
        \draw[fill=violetlight!30] (\x,0.2) rectangle (\x+0.6,0.8);
      }
      \node[below,font=\footnotesize] at (2,-0.3) {Salles à optimiser};
    \end{tikzpicture}
  \end{columns}
\end{frame}

% ====== SLIDE 23 : AVANTAGES DE L'APPROCHE ======
\begin{frame}{Avantages de l'Approche ACO}
  \begin{columns}
    \column{0.5\textwidth}
    \begin{block}{Flexibilité}
      \begin{itemize}
        \item Adaptation facile aux nouvelles contraintes
        \item Modification simple des poids
        \item Intégration de préférences
      \end{itemize}
    \end{block}
    
    \vspace{0.3cm}
    \begin{exampleblock}{Performance}
      \begin{itemize}
        \item Solutions de qualité
        \item Temps de calcul raisonnable
        \item Scalabilité
      \end{itemize}
    \end{exampleblock}
    
    \column{0.5\textwidth}
    \begin{alertblock}{Équilibre Optimal}
      \begin{tikzpicture}[scale=0.9]
        \draw[->,thick] (0,0) -- (4,0) node[right] {Temps};
        \draw[->,thick] (0,0) -- (0,3) node[above] {Qualité};
        
        % Courbe ACO
        \draw[ultra thick,cyandark] (0,0.5) .. controls (1,2) and (2,2.5) .. (4,2.8);
        \node[cyandark,right] at (4,2.8) {ACO};
        
        % Méthode exacte
        \draw[thick,neonpink,dashed] (0,0.5) -- (3.5,3);
        \node[neonpink,above] at (3.5,3) {Exacte};
        
        % Glouton
        \draw[thick,gray,dotted] (0,0.5) -- (1,1.2) -- (4,1.2);
        \node[gray,right] at (4,1.2) {Glouton};
      \end{tikzpicture}
    \end{alertblock}
  \end{columns}
  
  \vspace{0.5cm}
  \begin{block}{Robustesse}
    L'ACO évite les optimums locaux grâce au mécanisme d'évaporation et à l'exploration stochastique
  \end{block}
\end{frame}

% ====== SLIDE 24 : RÉSULTATS ATTENDUS ======
\begin{frame}{Résultats Attendus et Métriques}
  \begin{columns}
    \column{0.5\textwidth}
    \begin{alertblock}{Métriques Principales}
      \begin{enumerate}
        \item \textbf{Nombre de conflits}
        \[
          f_1(x) = \sum \text{conflits}
        \]
        
        \item \textbf{Taux d'occupation}
        \[
          \text{Occupation} = \frac{\sum E_i}{\sum Cap_j}
        \]
        
        \item \textbf{Équilibre charge}
        \[
          \sigma = \sqrt{\frac{1}{k}\sum(n_j - \bar{n})^2}
        \]
      \end{enumerate}
    \end{alertblock}
    
    \column{0.5\textwidth}
    \begin{block}{Critères de Succès}
      \begin{itemize}
        \item Zéro conflit hard
        \item Minimisation conflits soft
        \item Convergence rapide
        \item Stabilité solution
      \end{itemize}
    \end{block}
    
    \vspace{0.3cm}
    \begin{exampleblock}{Comparaison}
      Performance ACO vs :
      \begin{itemize}
        \item Algorithmes génétiques
        \item Recuit simulé
        \item Recherche taboue
        \item Méthodes exactes
      \end{itemize}
    \end{exampleblock}
  \end{columns}
\end{frame}

% ============================================
% SECTION 7 : CONCLUSION
% ============================================
\section{Conclusion et Perspectives}

% ====== SLIDE 25 : SYNTHÈSE ======
\begin{frame}{Synthèse de l'Étude}
  \begin{block}{Contributions Principales}
    \begin{enumerate}
      \item \textbf{Modélisation mathématique formelle} du problème de timetabling
      \item \textbf{Adaptation de l'ACO} avec heuristiques spécialisées
      \item \textbf{Intégration} contraintes hard et soft
      \item \textbf{Prise en compte} des disponibilités enseignants
    \end{enumerate}
  \end{block}
  
  \vspace{0.5cm}
  \begin{columns}
    \column{0.5\textwidth}
    \begin{alertblock}{Points Forts}
      \begin{itemize}
        \item Approche métaheuristique efficace
        \item Équilibre exploitation/exploration
        \item Flexibilité et adaptabilité
        \item Complexité maîtrisée
      \end{itemize}
    \end{alertblock}
    
    \column{0.5\textwidth}
    \begin{exampleblock}{Validation}
      \begin{itemize}
        \item Fondements biologiques solides
        \item Littérature académique riche
        \item Applications réussies
        \item Résultats prometteurs
      \end{itemize}
    \end{exampleblock}
  \end{columns}
\end{frame}

% ====== SLIDE 26 : PERSPECTIVES ======
\begin{frame}{Perspectives et Développements Futurs}
  \begin{columns}
    \column{0.5\textwidth}
    \begin{block}{Améliorations Algorithmiques}
      \begin{itemize}
        \item Hybridation avec recherche locale
        \item Parallélisation du calcul
        \item Apprentissage adaptatif paramètres
        \item Mécanismes de diversification
      \end{itemize}
    \end{block}
    
    \vspace{0.3cm}
    \begin{alertblock}{Extensions du Modèle}
      \begin{itemize}
        \item Multi-objectif (Pareto)
        \item Contraintes dynamiques
        \item Planification multi-semestre
        \item Optimisation énergétique
      \end{itemize}
    \end{alertblock}
    
    \column{0.5\textwidth}
    \begin{exampleblock}{Applications}
      \begin{itemize}
        \item Déploiement campus UNSTIM
        \item Interface utilisateur graphique
        \item Système d'aide à la décision
        \item Intégration SI existant
      \end{itemize}
    \end{exampleblock}
    
    \vspace{0.3cm}
    \begin{block}{Recherche Future}
      \begin{itemize}
        \item Analyse comparative approfondie
        \item Étude de sensibilité paramètres
        \item Validation sur benchmarks
        \item Publication résultats
      \end{itemize}
    \end{block}
  \end{columns}
\end{frame}

% ====== SLIDE 27 : RÉFÉRENCES CLÉS ======
\begin{frame}{Références Bibliographiques Clés}
  \begin{block}{Fondements de l'ACO}
    \begin{itemize}
      \item \textbf{Dorigo et al.} (1992, 1996) — Ant System original
      \item \textbf{Dorigo \& Stützle} (2004) — Ouvrage de référence
      \item \textbf{Stützle \& Hoos} (2000) — MAX-MIN Ant System
    \end{itemize}
  \end{block}
  
  \begin{block}{Applications au Timetabling}
    \begin{itemize}
      \item \textbf{Socha et al.} (2002) — MMAS pour timetabling universitaire
      \item \textbf{Mazlan et al.} (2019) — Modèle ACO avec contraintes
      \item \textbf{Aslan \& Aci} (2018) — Intégration disponibilités
    \end{itemize}
  \end{block}
  
  \begin{block}{Biologie et Éthologie}
    \begin{itemize}
      \item \textbf{Goss et al.} (1989) — Expérience du pont double
      \item \textbf{Bonabeau et al.} (2000) — Comportement insectes sociaux
    \end{itemize}
  \end{block}
\end{frame}

% ====== SLIDE 28 : CONCLUSION FINALE ======
\begin{frame}[plain]
  \begin{tikzpicture}[remember picture,overlay]
    \fill[left color=cyandark,right color=violetdeep] 
      (current page.south west) rectangle (current page.north east);
  \end{tikzpicture}
  
  \vspace{2cm}
  \begin{center}
    {\Huge\textcolor{white}{\textbf{Merci de Votre Attention}}}
    
    \vspace{1cm}
    {\Large\textcolor{cyanlight}{Questions \& Discussion}}
    
    \vspace{1.5cm}
    \begin{tikzpicture}
      \foreach \i in {1,...,8} {
        \fill[white,opacity=0.3] ({45*\i}:2) circle (0.15);
        \draw[white,opacity=0.5,thick] ({45*\i}:1.5) -- ({45*\i}:2.5);
      }
      \node[circle,fill=white,minimum size=1.5cm,opacity=0.9] at (0,0) 
        {\textcolor{cyandark}{\textbf{ACO}}};
    \end{tikzpicture}
    
    \vspace{1cm}
    {\large\textcolor{grisclair}{
      \textbf{Optimisation Continue et Discrète}\\
      UNSTIM — ENSGMM — 2025-2026
    }}
  \end{center}
\end{frame}

% ====== SLIDE BONUS : ALGORITHME DÉTAILLÉ ======
\begin{frame}[fragile,allowframebreaks]{Annexe : Algorithme Détaillé}
  \begin{algorithmic}[1]
    \State \textbf{Initialisation}
    \State $\tau_{ij} \leftarrow \tau_{max}$ pour tout $(i,j)$ compatible
    \State $S_{best} \leftarrow \emptyset$, $f_{best} \leftarrow +\infty$
    \State
    \While{non convergence}
      \State $S_{best\_cycle} \leftarrow \emptyset$
      \For{$ant = 1$ à $m$}
        \State $S^{ant} \leftarrow \emptyset$
        \State $C_{restantes} \leftarrow C$
        \While{$C_{restantes} \neq \emptyset$}
          \State Sélectionner $c_i \in C_{restantes}$
          \State $N_i \leftarrow$ salles compatibles pour $c_i$
          \For{chaque $r_j \in N_i$}
            \State Calculer $\eta^{cap}_{ij}$, $\eta^{conf}_{ij}$, $\eta^{load}_{ij}$
            \State $\eta_{ij} \leftarrow \eta^{cap}_{ij} \cdot (\eta^{conf}_{ij})^2 \cdot \eta^{load}_{ij}$
          \EndFor
          \State Calculer $p^{ant}_{ij}$ pour tout $j \in N_i$
          \State Sélectionner $r_{j^*}$ par roulette selon $p^{ant}_{ij}$
          \State $S^{ant} \leftarrow S^{ant} \cup \{(c_i, r_{j^*})\}$
          \State $C_{restantes} \leftarrow C_{restantes} \setminus \{c_i\}$
        \EndWhile
        \State Évaluer $f(S^{ant})$
        \If{$f(S^{ant}) < f_{best}$}
          \State $S_{best} \leftarrow S^{ant}$
        \EndIf
      \EndFor
      \State \textcolor{violetdeep}{\textbf{Mise à jour phéromones}}
      \For{tout $(i,j)$}
        \State $\tau_{ij} \leftarrow (1-\rho) \cdot \tau_{ij}$ \Comment{Évaporation}
      \EndFor
      \For{$(c_i, r_j) \in S_{best}$}
        \State $\tau_{ij} \leftarrow \tau_{ij} + Q/f(S_{best})$ \Comment{Dépôt}
      \EndFor
      \State Appliquer bornes MAX-MIN
    \EndWhile
    \State \Return $S_{best}$
  \end{algorithmic}
\end{frame}

\end{document}